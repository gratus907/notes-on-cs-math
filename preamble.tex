\usepackage[margin=0.65in, headsep=10pt, headheight=20pt,a4paper]{geometry}
% General Preamble template
\usepackage{amsmath, amssymb, amsfonts, mathtools, amsthm}
\usepackage{tocbasic}
\usepackage{esint}
\usepackage{physics}
\usepackage{footnote}
\usepackage{kotex}
\usepackage{enumerate}
\usepackage[bottom]{footmisc}
\usepackage{tabu}
\usepackage{authblk}
\usepackage{multicol}
\usepackage{float}
\usepackage{lipsum}
\usepackage{optidef}
\usepackage[hidelinks]{hyperref}
\usepackage{graphicx}
\usepackage[usenames,dvipsnames,svgnames]{xcolor}
\usepackage{dashrule}
\usepackage{tocloft}
\usepackage{makecell}
\usepackage{pdfpages}
\usepackage{listings}
\usepackage{titlesec}
\usepackage{lastpage}
\usepackage{changepage}
\usepackage[yyyymmdd,hhmmss]{datetime}
\newcommand{\floor}[1]{\left\lfloor #1 \right\rfloor}
\newcommand{\ceil}[1]{\left\lceil #1 \right\rceil}
\newcommand{\N}{\ensuremath{\mathbb{N}}}
\newcommand{\R}{\ensuremath{\mathbb{R}}}
\newcommand{\Z}{\ensuremath{\mathbb{Z}}}
\newcommand{\Q}{\ensuremath{\mathbb{Q}}}
\newcommand{\C}{\ensuremath{\mathbb{C}}}
\newcommand{\ord}{\text{ord}}
\newcommand{\di}{\mathrel{|}}
\usepackage{stmaryrd}
\newcommand{\contra}{\scalebox{1.5}{$\lightning$}}
\setlength{\parindent}{0em}
\theoremstyle{definition}

\makeatletter
\def\th@definition{%
  \thm@headfont{\bfseries}
  \thm@notefont{}% same as heading font
  \normalfont % body font
}
\makeatother
\everymath{\displaystyle}

\makeatletter
\def\thm@space@setup{%
  \thm@preskip=\parskip \thm@postskip=0pt
}



\def\currentlecture{Lecture}%
\newcommand{\lecture}[2]{
    \newpage
    \ifthenelse{\isempty{#2}}{%
      \renewcommand{\currentlecture}{Lecture #1}%
    }{%
      \renewcommand{\currentlecture}{Lecture #1: #2}%
    }%
    \pagebreak
    \section{\currentlecture}\label{\currentlecture}
}



% These are the fancy headers
\usepackage{fancyhdr}
\fancyhf{}
\pagestyle{fancy}

\renewcommand{\footrulewidth}{0.4pt}% Default \footrulewidth is 0pt
\makeatother


% Personal header
\newcommand{\legn}[2]{\left(\frac{#1}{#2}\right)}
\newcommand{\inner}[2]{\left\langle #1 , #2 \right\rangle}
\newcommand{\st}{\text{ such that }}
\newcommand{\for}{\text{ for }}
\newcommand{\Setcond}[2]{ \left\{\, #1 \mid #2 \, \right\}}
\newcommand{\setcond}[2]{\Setcond{#1}{#2}}
\newcommand{\seq}[1]{ \left\langle #1 \right\rangle}
\newcommand{\Set}[1]{ \left\{ #1 \right\}}
\newcommand{\set}[1]{ \set{#1} }
\newcommand{\halfline}{\vspace{0.5em}}
\newcommand{\infig}[2]{\begin{figure}[H]\centering\includegraphics[width=#1\linewidth]{#2}\end{figure}}
\newcommand{\infigcap}[3]{\begin{figure}[H]\centering\includegraphics[width=#1\linewidth]{#2}\caption{#3}\end{figure}}
\newenvironment{amatrix}[1]{%
  \left(\begin{array}{@{}*{#1}{c}|c@{}}
}{%
  \end{array}\right)
}
\def\numberline#1{}
\titleformat{\section}[block]{\Large\sffamily\bfseries}{\color{purple}\S\thesection}{1em}{}
\titleformat{\subsection}[block]{\large\sffamily\bfseries}{\color{purple}\S\thesubsection}{1em}{}
\renewcommand{\cftsecfont}{\bfseries\sffamily}
\renewcommand{\cftsubsecfont}{\sffamily}
\raggedbottom


\def\ScribeStr{Wonseok Shin}
\def\LecStr{??}
\def\LecNum{??}
\def\LecTitle{??}
\def\LecDate{??}
\def\CorTitle{??}
\def\CorInfo{??}
\newcommand{\Course}[1]{\def\@course{#1}}
\newcommand{\Scribe}[1]{\def\ScribeStr{Scribe: #1}}
\newcommand{\Scribes}[1]{\def\ScribeStr{Scribes: #1}}
\newcommand{\Lecturer}[1]{\def\LecStr{Lecturer: #1}}
\newcommand{\Lecturers}[1]{\def\LecStr{Lecturers: #1}}
\newcommand{\LectureNumber}[1]{\def\LecNum{#1}}
\newcommand{\LectureDate}[1]{\def\LecDate{#1}}
\newcommand{\LectureTitle}[1]{\def\LecTitle{#1}}
\newcommand{\CourseTitle}[1]{\def\CorTitle{#1}}
\newcommand{\CourseInfo}[1]{\def\CorInfo{#1}}
\newdimen\headerwidth

\newcommand\invisiblesection[1]{%
\refstepcounter{section}%
\addcontentsline{toc}{section}{\protect\numberline{\thesection}\LecDate : #1}%
\sectionmark{#1}}

\newcommand{\NewLecture}{
\noindent
\thispagestyle{empty}
\begin{center}
  \framebox{
    \vbox{
      \sffamily
      \headerwidth=\textwidth
      \advance\headerwidth by -0.22in
      \hbox to \headerwidth {{\CorTitle \hfill [\CorInfo]\hspace{-0.2em}}}
      \vspace{4mm}
      \hbox to \headerwidth {{\bfseries\Large \hfill Lecture \LecNum: {\LecTitle} \hfill}}
      \vspace{2mm}
      \hbox to \headerwidth {\hfill \LecDate \hfill}
      \vspace{2mm}
      \hbox to \headerwidth {{\LecStr \hfill Scribe : \ScribeStr}}
      \invisiblesection{\LecTitle}
      }
    }
\end{center}
\vspace*{4mm}}




\usepackage{thmtools}
\usepackage[framemethod=TikZ]{mdframed}
\mdfdefinestyle{mdblackbox}{%
    roundcorner = 8pt,
    linewidth=1pt,
    skipabove=12pt,
    innerbottommargin=9pt,
    skipbelow=2pt,
    linecolor=black!30!,
    nobreak=true,
	backgroundcolor=RedViolet!5!gray!5,
}
\declaretheoremstyle[
    headfont=\sffamily\bfseries\color{black},
    mdframed={style=mdblackbox},
    headpunct={\\[3pt]},
    postheadspace={0pt}
]{thmblackbox}
\mdfdefinestyle{mdredbox}{%
    roundcorner = 8pt,
	linewidth=0.5pt,
	skipabove=12pt,
	frametitleaboveskip=5pt,
	frametitlebelowskip=0pt,
	skipbelow=2pt,
	frametitlefont=\bfseries,
	innertopmargin=4pt,
	innerbottommargin=8pt,
	nobreak=true,
	backgroundcolor=Salmon!5,
	linecolor=RawSienna,
}
\declaretheoremstyle[
	headfont=\sffamily\bfseries\color{RawSienna},
	mdframed={style=mdredbox},
	headpunct={\\[3pt]},
	postheadspace={0pt},
]{thmredbox}
\mdfdefinestyle{mdbluebox}{%
	roundcorner = 8pt,
	linewidth=1pt,
	skipabove=12pt,
	innerbottommargin=9pt,
	skipbelow=2pt,
	linecolor=blue!30,
	nobreak=true,
	backgroundcolor=TealBlue!5,
}
\declaretheoremstyle[
	headfont=\sffamily\bfseries\color{MidnightBlue},
	mdframed={style=mdbluebox},
	headpunct={\\[3pt]},
	postheadspace={0pt}
]{thmbluebox}
\declaretheorem[style=thmblackbox,name=Theorem,numberwithin=section]{theorem}
\declaretheorem[style=thmblackbox,name=Lemma,sibling=theorem]{lemma}
\declaretheorem[style=thmblackbox,name=Proposition,sibling=theorem]{proposition}
\declaretheorem[style=thmblackbox,name=Corollary,sibling=theorem]{corollary}
\declaretheorem[style=thmbluebox,name=Definition,sibling=theorem]{definition}
\declaretheorem[style=thmblackbox,name=Theorem,numbered=no]{theorem*}
\declaretheorem[style=thmblackbox,name=Lemma,numbered=no]{lemma*}
\declaretheorem[style=thmblackbox,name=Proposition,numbered=no]{proposition*}
\declaretheorem[style=thmblackbox,name=Corollary,numbered=no]{corollary*}
\declaretheorem[style=thmblackbox,name=Algorithm,sibling=theorem]{algorithm}
\declaretheorem[style=thmblackbox,name=Algorithm,numbered=no]{algorithm*}
\declaretheorem[style=thmblackbox,name=Claim,sibling=theorem]{claim}
\declaretheorem[style=thmblackbox,name=Claim,numbered=no]{claim*}
\declaretheorem[style=thmredbox,name=Example,sibling=theorem]{example}
\declaretheorem[style=thmredbox,name=Example,numbered=no]{example*}
\declaretheorem[style=thmblackbox,name=Remark,sibling=theorem]{remark}
\declaretheorem[style=thmblackbox,name=Remark,numbered=no]{remark*}

\theoremstyle{definition}
\newtheorem{conjecture}[theorem]{Conjecture}
\newtheorem{fact}[theorem]{Fact}
\newtheorem{answer}[theorem]{Answer}
\newtheorem{ques}[theorem]{Question}
\newtheorem{exercise}[theorem]{Exercise}
\newtheorem{problem}[theorem]{Problem}
\newtheorem*{conjecture*}{Conjecture}
\newtheorem*{definition*}{Definition}
\newtheorem*{fact*}{Fact}
\newtheorem*{answer*}{Answer}
\newtheorem*{ques*}{Question}
\newtheorem*{exercise*}{Exercise}
\newtheorem*{problem*}{Problem}
